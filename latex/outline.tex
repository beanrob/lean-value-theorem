\documentclass{article}
\usepackage{amsfonts}
\usepackage{parskip}
\usepackage[a4paper, margin=1cm]{geometry}
\pagenumbering{gobble}
\title{\textbf{Formalisation Outline}}
\author{Lean Value Theorem \\
\normalsize{Odysseas Gabrielatos, Chris Levesley, Ben Robinson, Saumya Shah}}
\date{November 2025}

\begin{document}
\maketitle

\section{Formalisation Details}

Our group intends to formalise the Lagrange's Mean Value Theorem:

\smallskip

\begin{center}
Let $f: [a, b] \to \mathbb{R}$ be continuous, and differentiable on $(a, b)$.
Then $\exists \, c \in (a, b)$ such that $f'(c) = \frac{f(b) - f(a)}{b
- a}$.
\end{center}

\smallskip

For the rest of the document, assume that Mean Value Theorem/MVT refers to
Lagrange's version.

\subsection{Assumptions \& Imports from MathLib}

As a starting point, we will import the definition of the real numbers and
functions. However, from there we will take the steps to proving the Mean Value
Theorem ourself, including formalising notions of continuity and derivatives.

If it turns out that this does not prove for a complex enough project, we may
turn back this assumption and implement the real numbers ourselves. Equally, we
may extend the project beyond Lagrange's Mean Value Theorem if we need to,
likely at least to Cauchy's version of the theorem, and possibly further into
analysis. This makes the project very flexible with regard to timings and
deliverables.

We are not currently sure how many tactics will be imported from MathLib - it is
very unlikely that the entirety of MathLib.Tactic will be necessary for the
project, so it is probable that we will begin by importing the entire tactics
library, then reducing the scope of our imports later into the project when we
have a better idea of the "shape" of proofs we wish to write.

\subsection{Formalisation Route}

The main necessary step to proving the Mean Value Theorem is proving Rolle's
Theorem, since the former is a generalisation of the latter. The MVT should
follow somewhat naturally from proving Rolle's Theorem and should require very
little additional work after it is implemented.

Because our starting point is the definition of the real numbers, the primary
steps we will need to take to formalise the Mean Value Theorem are as follows:

\begin{itemize}
    \item Intervals, limits and algebra of limits
    \item Continuity, derivatives and differentiation
    \item Rolle's Theorem
    \item Mean Value Theorem
\end{itemize}

\subsection{Compared to the MathLib Version}

Compared to the version in MathLib, our formalisation, of course, will only be
for the real numbers. Due to the nature of MathLib, several of the building
blocks for the MVT are defined for much more generic spaces. We hope that our
restriction to the real numbers will allow for these building blocks to be able
to be rewritten in a more intuitive and readable manner thanks to the statements
hopefully being less clunky due to the reduced scope.

\section{Project Management}

We will distribute the work amongst team members evenly by having each person
working on one step of the process at once. A group meeting will be held where
we plan a framework for the code, with each theorem or lemma being proved by
means of a "sorry" statement. This allows for people to work on formalising
statements even if their assumptions are not yet ready.

All work will be shared remotely via Git to a remote repository on GitHub to
ensure all members of the group have up-to-date code. Additionally, GitHub's
built-in features such as Issues and Projects will be used to manage what needs
to be done and prioritise the implementation of certain things. In particular,
the Issues feature will be used to assign particular pieces of work to each
person, so we all know who is working on what.

\end{document}
